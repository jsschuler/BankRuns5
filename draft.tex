 %
%  untitled
%
%  Created by John Schuler on 2016-11-14.
%  Copyright (c) 2016 __MyCompanyName__. All rights reserved.
%
\documentclass[12pt]{article}
\newcommand\tab[1][1cm]{\hspace*{#1}}
% Use utf-8 encoding for foreign characters
\usepackage[utf8]{inputenc}
\usepackage{float}
\floatstyle{boxed}
\restylefloat{figure}
\usepackage{caption}
% Setup for fullpage use
\usepackage{fullpage}
\usepackage[table,xcdraw]{xcolor}
\usepackage{booktabs}
\usepackage{tikz}
% Uncomment some of the following if you use the features
%
% Running Headers and footers
%\usepackage{fancyhdr}

% Multipart figures
%\usepackage{subfigure}

% More symbols
%\usepackage{amsmath}
%\usepackage{amssymb}
%\usepackage{latexsym}

% Surround parts of graphics with box
\usepackage{boxedminipage}
\newcommand{\dummyfigure}{\tikz \fill [blue] (0,0) rectangle node [black] {Figure} (2,2);}
% Package for including code in the document
\usepackage{listings}
\usepackage{amsmath}
\usepackage{amssymb}
% If you want to generate a toc for each chapter (use with book)
\usepackage{minitoc}

% This is now the recommended way for checking for PDFLaTeX:
\usepackage{ifpdf}
\usepackage{natbib}
\usepackage{wrapfig}
%\usepackage[numbers]{natbib}
%\newif\ifpdf
%\ifx\pdfoutput\undefined
%\pdffalse % we are not running PDFLaTeX
%\else
%\pdfoutput=1 % we are running PDFLaTeX
%\pdftrue
%\fi

%\ifpdf
%\usepackage[pdftex]{graphicx}
%\else
%\usepackage{graphicx}
%\fi
\title{A Bank Run Model for the Twentieth Century}
\author{John S. Schuler \\ Department of Computational and Data Sciences \\ George Mason University}

\date{\today}
\usepackage{setspace}
\usepackage{Sweave}
\begin{document}
\Sconcordance{concordance:draft.tex:draft.Rnw:1 65 1 1 0 97 1}

\maketitle
\begin{abstract}
	Diamond and Dybvig 1983 is a now classic model of banking failure. This model and the considerable ancilliary literature studies two equilibria: the ``good'' equilibrium of bank stability and the ``bad'' equilibrium of bank failure. A major limitation of these models is that while they acknowledge the fact of these two equilibria, they are silent on how a system in the desired equilibrium suddenly moves into the run equilibrium. Agentization refers to the process of taking usually classic models, economic or otherwise, and representing them in agent-based simulations that hopefully reproduce those model's central features. I consider an agentized Diamond-Dybvig model that reveals some major conceptual limitations in Diamond-Dybvig that limits its utility as the foundation of agent-based studies of bank runs. Then I present an alternative bank run model that may provide such a basis not only for the study of bank runs but also for broader models of financial contagion.
\end{abstract}
\begin{doublespace}
\end{doublespace}



\end{document}
